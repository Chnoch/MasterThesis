
\documentclass[oneside,a4paper]{book}
%\pagestyle{headings}

\input{preamble}

% A B S T R A C T
% % % % % % % % % % % % % % % % % % % % % % % % % % % % % % % % % %
\chapter*{\centering Abstract}
\begin{quotation}
\noindent 
Abstract (max. 1 page)

Name of the Supervisor, Group, Institute, University, Supervisor

Name of the Assistant, Group, Institute, University, Assistant
\end{quotation}
\clearpage


% C O N T E N T S 
% % % % % % % % % % % % % % % % % % % % % % % % % % % % % % % % % % % % % % % %
\tableofcontents


\newpage
\section{Abstract}

In this thesis we apply different techniques of machine learning to a real life dataset, gathered anonymously from users of an Android app that collects data about the stations the user has passed. Our goal is to best predict the next station that the user is going to, based on information that would be available to us in the moment. Such information could include time, day, current location, past locations.

We show ways to analyze and optimize the raw data, prepare and filter it in a way that yields the best results for further analysis. We compare different machine learning algorithms on our dataset and discuss the advantages and disadvantages that we encountered with each algorithm. We then present the experiments that we did on our dataset and evaluate the results and predictions from the algorithms to draw conclusions to the reliability and the meaningfulness of our approaches. We further discuss what future work should or could be done to take advantage of the techniques that we explored in our thesis.

\newpage
\section{Introduction}
Machine Learning has seen an incredible boost of interest in recent years in the research community as well as in industrial applications. Due to the amount of data being generated and gathered across various disciplines new approaches to data analysis have to be discovered. It is no longer feasible and sufficient to manually skim through the data and draw conclusions from these analysis. The process would usually be too slow and the amount of data too overwhelming to have a reasonable process of analysis. This issue has led to the rising of new machine learning techniques and related fields which fall under the broader term of artificial intelligence (AI). Many different algorithms and techniques have been developed to account for different problems, data sets and procedures.

The main goal of our thesis was to evaluate and compare different approaches and machine learning algorithms on an existing data sets. We also evaluate different types of use cases and combinations of features that are available in our data sets. We compare the results of these evaluations and try to find a combination of data preparations, algorithms and evaluations that yield results successful enough to be able to be used in an actual application. 

We have gathered geographical and temporal data of users of an Android application that is used for looking up public transportation information. With the data set we should be able to create anonymized user profiles, the precision of which always reflecting the amount of data we have on each user. Based on the historical information we have about each user we try to predict the future behavior of the user, based on features such as current station, day of the week, time, past station(s). Our goal is to accurately predict the next station the user will go to and therefore be able to support the user in ordinary day to day tasks. 

An example use case would be to create a personal assistant that "knows" a users daily routine from historical data and without the user having to explicitly state it. Combined with other data sources such as geographical locations (home, work, gym, train stations, shopping centers, etc.), calendar entries or current public transport timetables the assistant could provide information about when to leave the current place in order to get to the next place that is to be targeted in the users routine. 

The part of this thesis in such a use case is to provide the underlying machine learning framework that can be included in such an application. For that we take the available data we have and analyze the necessary steps to get predictions as accurate as possible. The conclusion drawn from this thesis can be integrated in a more comprehensive application that covers different use cases. We will cover this in more detail in the chapter about future work (section~\ref{sec:future_work}).

\newpage
\section{Related Work}


Mining temporal patterns of transport behaviour for predicting future transport usage\\
Stefan Foell, Gerd Kortuem; Reza Rawassizadeh; Santi Phithakkithukoon; Marco Veloso and Carlos Bento\\
Third International Workshop on Pervasive Urban Applications, 8 Sep 2013, Zurich, Switzerland

Challenges in Spatial-Temporal Data Analysis Targeting Public Transport\\
Mohammad Sajjad Ghaemi, Bruno Agard, Vahid Partovi Nia, Martin Trépanier\\
Polytechnique Montrèal, 2015
\section{Sequence Prediction}

Here goes the Sequence description.

\subsection{Task Description}

(we model our problem as a sequence prediction problem)


\subsection{Proposed Solutions}
(Detailed Description of Different Classifiers used, and how they are going to be used)
\section{Experiments and Evaluation}

Here goes the experiments.


\subsection{Introduction / Procedure}

\subsection{Data Set / Data Analysis}

\subsection{Naive Approach}


\subsection{Machine Learning Results}

\subsubsection{Execution Plan}
\subsubsection{Data Preparation / Statistical Analysis}
\subsubsection{Decision Trees (Random Forest is included)}
\subsubsection{Naive Bayes}
\subsubsection{HMM / Neural Networks}

\subsection{Comparison of Results}
\newpage
\chapter{Conclusion}

One of the most important goals we wanted to achieve was to find a way to reliable predict the public transport behavior of our users. Given the limitations we placed upon the data necessary for the task we have achieved a respectable result. We have been able to run our machine learning algorithms for 150 - 250 users, depending on the use case, and feel comfortable of using the predicted results of at least 100 - 150 users. While this is a good start, the amount of users that could benefit of our system is still quite small. We have run our predictions for 5-10\% of our user base. This is mostly influenced by the missing amount of data. Over 75\% of our users have less than ten distinct data points gathered. This is in no way enough data to successfully make a prediction and be convinced about it enough to present it to our user. For the vast majority of users the data gathering would need to be optimized before we can think about including them into our machine learning system.

However for the users that have enough good quality data and that can be evaluated we have seen a good overall prediction. For most algorithm and feature set combination we have a mean F-Measure close to 0.9. If we'd cut off the lowest 10\% of our predicted users for which the reliability of the predictions is too low, we'd get an even higher F-Measure, close to 0.95. This is a very good starting point and with direct feedback from the user as to whether our predictions match the ground truth it could be improved even faster.

We have seen that while the Multilayer Perceptron was by far the slowest of all algorithms, it also performed best. It had a slightly better overall prediction and was also more stable than the decision tree algorithm and the naive Bayes. Precision and recall in general was very similar, but if computing costs are no issue then the multilayer perceptron outperforms the other algorithms.

We have successfully been able to achieve the goal we set out to, creating a useful and integrable way of predicting user behavior based on historical data of public transport usage of these users.
\section{Future Work}
\label{sec:future_work}

Here goes the future work.
%\chapter {Related Work}
%In which we learn what have other done to address similar problems. For example, the work of Star \cite{Star89}
%
%\chapter{The Problem}
%In which we understand what the problem is in detail.
%
%\chapter {The Solution}
%In which you describe your solution.
%
%\chapter {The Validation}
%In which you show how well the solution works.
%
%\chapter {Conclusion and Future Work}
%In which we step back, have a critical look at the entire work, then conclude, and learn what lies beyond this thesis.



%END Doc
%-------------------------------------------------------

\bibliography{thesis}
\bibliographystyle{plain}

\end{document}
