\newpage
\section{Abstract}

In this thesis we apply different techniques of machine learning to a real life dataset, gathered anonymously from users of an Android app (Farplano) that collects data about the stations the user has passed on his trip. Our goal is to get an as accurate as possible prediction of the next station that the user is going to, based on information that is available to us in the moment. This can include the current time, day, location and past locations of the current trip or previous trips.

As raw data we have gathered data points that include the user id, the current timestamp and the station id of the closest station. We have first coarsely optimized our raw data set, prepared and filtered it in a way to yield the best results for further analysis. Before doing further optimizations we have analyzed our data set based on different viewpoints and used a baseline approach to get a first test of the accuracy. On that baseline approach we have achieved 48\% of correctly predicted data entries.

We applied statistical analysis to further refine and reduce our data set. We have removed low-profile users and have reduced the amount of stations that we want to be able to predict. We compared multiple machine learning algorithms and analyzed the advantages and disadvantages of each of the algorithms. We executed different experiments with a combination of feature sets with all algorithms and evaluated the results and predictions to draw conclusions to the reliability and the meaningfulness of our approach. We achieved an average accuracy on our predictions of 92\%. We further discuss what future work can be done to take advantage of the techniques that we explored in our thesis.
