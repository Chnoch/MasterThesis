\newpage
\chapter{Introduction}
The main goal of our thesis is to accurately predict future behavior of users of our Android application based on historical data we have gathered about each user. We want to be able to support the user in ordinary day to day tasks and to provide assistance for the daily routine. We have gathered geographical and temporal data of users of an Android application that is used for looking up public transportation information. With the data set we should be able to create anonymized user profiles, the precision of which always reflecting the amount of data we have on each user. With this data we try to predict future behavior based on features such as current station, day of week, time and past station(s). We haven't integrated the predictions into the Android application for the end user, but we provide an analysis of the steps that are needed and have evaluated different implementations and their respective prediction accuracy. The implementation we created can be directly used for the integration into the Android application, given some fine tuning of the parameters.

To achieve our main goal we first analyzed our data set, comparing temporal and geographical properties of each user in different combinations. This allowed us to get an understanding of how our data is structured and the possible pitfalls and issues we might encounter. We have been able to include the conclusions from this analysis into account when preparing the data for the machine learning toolkit. The baseline approach we created served as a starting point and a reference for our predictions. We then evaluated and compared different approaches and machine learning algorithms on our data set. We also evaluated different types of use cases and combinations of features available in our data set. We compared the results of these evaluations and tried to find a combination of data preparations, algorithms and evaluations that yielded results successful enough to be able to be used in an actual application. 

Machine learning in general has seen an incredible boost of interest in recent years in the research community as well as in industrial applications. Due to the amount of data being generated and gathered across various disciplines new approaches to data analysis had to be discovered. It is no longer feasible and sufficient to manually skim through the data and draw conclusions from these analysis. The process is usually too slow and the amount of data too overwhelming to have a reasonable process of analysis. This issue has led to the rising of new machine learning techniques and related fields which fall under the broader term of artificial intelligence (AI). Many different algorithms and techniques have been developed to account for different problems, data sets and procedures. We take advantage of the research that has gone into machine learning and use a toolkit called WEKA (explained in more detail in chapter \ref{subsection:WEKA}). WEKA provides a framework with lots of different implementations of machine learning classifiers and tools that aid in the preparation, execution and evaluation of our gathered data.

An example use case of the results of our thesis would be a personal assistant that "knows" a users daily routine from historical data and without the user having to explicitly state it. Combined with other data sources such as geographical locations (home, work, gym, train stations, shopping centers, etc.), calendar entries or current public transport timetables the assistant could provide information about when to leave the current place in order to get to the next place that is to be targeted in the users routine. 

The part of this thesis in such a use case is to provide the underlying machine learning framework that can be included in such an application. For that we take the available data we have and analyze the necessary steps to get predictions as accurate as possible. The conclusion drawn from this thesis can be integrated in a more comprehensive application that covers different use cases. We will cover this in more detail in the chapter about future work (section~\ref{cha:future_work}).
