\newpage
\section{Introduction}
Machine Learning has seen an incredible boost of interest in recent years in the research community as well as in industrial applications. Due to the amount of data being generated and gathered across various disciplines new approaches to data analysis have to be discovered. It is no longer feasible and sufficient to manually skim through the data and draw conclusions from these analysis. The process would usually be too slow and the amount of data too overwhelming to have a reasonable process of analysis. This issue has led to the rising of new machine learning techniques and related fields which fall under the broader term of artificial intelligence (AI). Many different algorithms and techniques have been developed to account for different problems, data sets and procedures.

The main goal of our thesis was to evaluate and compare different approaches and machine learning algorithms on an existing data sets. We also evaluate different types of use cases and combinations of features that are available in our data sets. We compare the results of these evaluations and try to find a combination of data preparations, algorithms and evaluations that yield results successful enough to be able to be used in an actual application. 

We have gathered geographical and temporal data of users of an Android application that is used for looking up public transportation information. With the data set we should be able to create anonymized user profiles, the precision of which always reflecting the amount of data we have on each user. Based on the historical information we have about each user we try to predict the future behavior of the user, based on features such as current station, day of the week, time, past station(s). Our goal is to accurately predict the next station the user will go to and therefore be able to support the user in ordinary day to day tasks. 

An example use case would be to create a personal assistant that "knows" a users daily routine from historical data and without the user having to explicitly state it. Combined with other data sources such as geographical locations (home, work, gym, train stations, shopping centers, etc.), calendar entries or current public transport timetables the assistant could provide information about when to leave the current place in order to get to the next place that is to be targeted in the users routine. 

The part of this thesis in such a use case is to provide the underlying machine learning framework that can be included in such an application. For that we take the available data we have and analyze the necessary steps to get predictions as accurate as possible. The conclusion drawn from this thesis can be integrated in a more comprehensive application that covers different use cases. We will cover this in more detail in the chapter about future work (section~\ref{sec:future_work}).
