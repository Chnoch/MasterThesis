\newpage
\section{Future Work}
\label{sec:future_work}

Even though we have successfully completed our experiments and have, under certain restrictions to our data, gathered a respectable precision with our predictions, there are still things left to be improved. In this final section we list the tasks that could and / or should be taken up following this thesis.

As a logical next step the approaches that we have implemented could be included in the actual Android application where the data is gathered. If carefully and helpfully integrated into the application so as that the user actually benefits of it but doesn`t consider it as too intruding in his privacy or too imprecise in the predictions this could lead to a virtuous cycle. The more use the user has the more he uses the feature, in turn providing our system with more data which generally leads to a better prediction for the user. The introduction of the feature would need to carefully monitored, so that it is only active for user with enough gathered data that the system is confident enough of providing a good estimate. Otherwise the prediction might be too shaky, leaving the user with little benefit and generally loosing prospective users in the process. It might also be necessary to do some A/B tests to figure out what the target precision for a prediction needs to be in order to display it to the end user. Depending on how much or how often the user likes to see recommendations the precision could be more or less important and the behavior of the application can be adapted.

In combination with creating an actual integration a more flexible approach to the statistical analysis and data preparation and also to the options of the machine learning algorithms might be taken. After our analysis phase and some trials on the data we have imposed relatively fixed limits on what conditions our data needs to have in order to be relevant for our analysis. Depending on how the user actually behaves it might make sense to either lift or to further limit certain restrictions, based on the complexity that the users data profile inhibits. The overall precision of the system might not be improved too much, however for certain edge cases this could lead to significant gains or could successfully delay the integration of the system until enough data and information is gathered to be precise enough. 

Another type of prediction that we could try to establish is the current position of the user. We have in our whole thesis only ever predicted where the user might go next, however it could also be interesting to get a best guess for the current location of the user, based on time, day, previous usages of the app, and so on. This might help in getting a faster coarse location, faster than GPS and cell signal triangulation. If the predictions would be precise it could easily lead to a great contribution for the Android application.

On a technical level there are multiple options to go forward. One would be to include a form of global state for the prediction. We have only ever looked at a users data profile in isolation. It might be interesting to evaluate certain patterns that are valid for all users and take these into account when creating a prediction for a single user. This might especially be helpful if the system encounters a case where previous data from the user doesn`t yield a probable prediction. Since humans tend to be herd animals and follow similar patterns this might also be a helpful contribution.

Another technical option might be to switch the machine learning framework for something more recent. A good example to evaluate would be Google`s TensorFlow, an open source framework based on Python that is founded on the years of research that Google has put into machine learning. TensorFlow is mostly based on neural networks and can be scaled from anything from a smartphone to a large-scale datacenter. It might be very interesting to see what advantages can be gained using an absolute state-of-the-art framework that is already being used in applications by billions of users. The exact effect it would have on our problem set is hard to predict, but it could very well lead to an improved predictions while making less restrictions on the data. 