\newpage
\chapter{Related Work}

As mentioned in the Introduction our goal is to successfully predict a users future public transport behavior. To achieve that goal we will first look at the problem description in detail in the next chapter. We will then discuss the experiments and evaluations that we did. We will show the analysis we have done for our data, the preparation we have made and the experiments we executed before we come to the conclusion

\section{Data Analysis}
We have made an extensive analysis over our data set to get an idea of the way our data is distributed, as well as the advantages and challenges that we might face. We have compared our analysis to the work presented by Mohammad Sajjad Ghaemi et al. in \cite{RelatedWorkDataAnalysis}. 

They have specifically looked at challenges in spatial-temporal data analysis targeting public transport. They have investigated problems and challenges that might arise doing large scale data analysis and sequence prediction on top of publicly available usage data of public transport. They have reviewed these scenarios and suggested solutions to take the different problems into account. While they have only worked on a theoretical base for the analytic problem that could be faced, we have done our analysis mainly on the actual data we have at hand. Since our analysis was only a first step on the way we have made our analysis on a more granular level. We have already anticipated that we won't be able to implement a highly tuned and fine-grained system. 

Most scenarios analyzed in \cite{RelatedWorkDataAnalysis} are targeting spatial behavior, the way users use public transport and the trips they take meanwhile. The reasoning behind their analysis is to figure out similar patterns that users have. The assumptions that users with similar usage patterns will also in the future behave similarly can lead to a significant increase in prediction precision. Most usages between users might however not be the exact same sequence of of stations, but a path that bears certain similarities. Scenarios they have listed includes users with the same starting point and ending point, users taking the same buses in opposite directions, users with same directional patterns or same symmetric direction patterns, however not on the exact same stations. Other more complex use cases are the same pattern except for one or two outliers, a shared subsequence of stations or the same resulting distance of travel. They even went further and analyzed similarities of patterns based on a circular grid representation of bus stops or of a pairwise bus stop similarity coefficient. For all the scenarios they have analyzed the possible benefits of using the scenario and possible solutions to incorporate them. 

As for temporal data they have presented a distance calculation technique based on the k-means clustering method. In this method the temporal data of a user is encoded as a 0-1 vector. When comparing these n-dimensional vectors between users their similarity can be computed quite easily. This yields a fast and simple way of analyzing similar temporal usage patterns between users.

As written above they have created a far more detailed analysis than was suitable for our thesis. We have also been looking more at point based analytics (e.g. how often do users use a station) and less trip-based analysis, which is done in most of the spatial analysis in \cite{RelatedWorkDataAnalysis}.

However it has still given us a good overview of the challenges lying ahead and the possible stumbling (or stepping) stones that lie in more targeted data analysis.


\section{Problem Description}
To aid us in describing our problem and to serve as a source on the theoretical basics of machine learning in general and many different algorithms in specific as well as a good overview over WEKA we have made extensive use of \cite{DataMining}. The book written by Ian H. Witten, Eibe Frank and Mark A. Hall served as a comprehensive reference for any theoretical questions of machine learning techniques as well as on WEKA. 

\section{Experiments and Evaluations}
As will be shown later we have made our experiments and evaluations on a combination of temporal and spatial features gathered from our data. We have analyzed research done in similar areas and how they compare to the work we have done. We have specifically analyzed a paper by Stefan Foell et al., experimenting with mining temporal patterns of transport behavior for predicting future transport usage \cite{RelatedWorkTemporalPatterns}. Their goal was to predict whether a user would use public transport on a given day in the future, based on his previous usage patterns.

They have used anonymized data from automated fare collection systems from bus rides in Lisbon to train and test their approach. The highest prediction accuracy they have achieved in their tests is 77\%. As a comparison they have included two simple baseline approaches and they have achieved a far higher accuracy using machine learning techniques. The main difference to our approach is that they have done their experimentation on a much coarser level. We have tried to predict single future trips and not simply the overall day-to-day public transport behavior.

For the actual prediction they have created four different temporal features of their data set. The first is the part of the week, being encoded as either weekday or weekend. The second feature is the day of the week. We have used both of these features as well in our experiments. In addition they have used the travel periodicity of a user, specifying the typical time periods that underlie the user's access to the buses. They also include the travel stationarity, specifying how long a user has been continuously using public transport before the to be predicted day. These are features that we haven't included as the experiments we did were on a more fine grained level and we have thus focused more on the specific time of day in our features.

Using these features they have trained a naive Bayes algorithm with a 80\% - 20\% split of their data set. They have used a superset of their features for the prediction, leading to lots of combinations in feature sets. Their accuracy fluctuates between 0.655 and 0.774. The accuracy they have achieved is a lot lower than what we were able to predict, however we believe that this is mostly due to the statistical preparation we have done. They have used their complete data set which is a lot larger than what we have used and they haven't put any limitations on their data set.